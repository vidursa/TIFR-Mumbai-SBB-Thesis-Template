\chapter[Acknowledgments]{\centering Acknowledgments}

%\Blindtext 

In the culmination of this arduous journey, as I stand on the threshold of completing my PhD thesis, I am filled with a profound sense of gratitude towards the individuals who have been instrumental in shaping this endeavor. While this thesis originates from me, countless individuals have gone on to shape it whether knowing or unknowingly. I am honored to acknowledge the contributions for a few of these individuals who have extended their support, and act as inspiration to propel me forward.

First and foremost, I extend my heartfelt gratitude to my advisor, Dr. Sandhya Koushika. Your mentorship has been invaluable throughout this journey. Your insightful guidance, and dedication to academic excellence have been a guiding light for me. Your ability to challenge my ideas and encourage critical thinking has been pivotal in shaping the direction of my thinking. I am fortunate to have had the opportunity to work under your guidance. I would also like to thank the departmental faculty members Dr. Krishanu Ray, Dr. Ullas Kolthur and Dr. Mahendra Sonawane for their critical input and guidance throughout my journey. In addition, I would like to thank Dr. Aprotim Mazumder and Dr. K. Subramaniam for their inputs and hospitality during my stay at TIFR-Hyderabad and IIT-Madras respectively. I am also fortunate to have wonderful collaborators Dr. Debasish Chaudhuri, Dr. Amitabha Nandi and Dr. Amir Shee. Their input and multiple discussions have made my scientific arguments and communication better.

Within the lab, I am deeply grateful to Dr. Parul Sood, who had been a major reason for me joining this lab. Her work ethic, organization, unwavering motivation and ability to get stuff done has been a guiding post for my personal development and I will always cherish this mentorship. I would also like to express my deep appreciation to Dr. Neena Ratnakaran and Anusheela Chatterjee for their support and input during my initial years in TIFR. Both Neena and Anusheela have been instrumental at giving suggestions and reforming my lab ethic. Your keen eye for detail and commitment to excellence have pushed me to strive for clarity and precision in my thinking and writing. In addition, I would like to thank Anusheela for introducing me to the world of science outreach, where I have learnt a lot of soft skills as well as made a lot of lifelong friends. I would also like to thank Dr. Sucheta Kulkarni, who I have worked with in my initial years in the lab. While working on the paper together, it was really helpful to learn how to conduct myself professionally with collaborations along with how to write academically, and coordinate multiple experiments with a work-life balance. These experiences are truly invaluable.

Within the outreach community, I would like to express my gratitude towards Dr. Arnab Bhattacharya, who has been an exemplary guide to communicate tough ideas succinctly with the public. Along with Arnab, I am deeply grateful to Surendra Kulkarni, together who have shown me how to manage a large group motivated towards a common goal. Kulkarni ji has also been a great example of having a fearless attitude to try novel approaches to communicate and to work with different kinds of people to make an event successful. Your guidance and support throughout the years have truly enriched my experience within TIFR. On this note, I would also like to thank Dr. Shubha Tole. While working on many departmental outreach events, I have learnt a great deal from observing you. I have tremendous respect for your abilities as a leader and aim to strive towards having incisive decisions, crystal clear communication, agency for action, and ability to gather and motivate a group. I would also like to thank Mayank Narang and Lankeshwar Dey, together with whom I have organized and handled multiple events. It was a pleasure to work with you both.

I would also like to thank my batchmates, Komal, Namrata, and Priya who have always been there alongside me having constant discussions, helping refine each other thoughts and taking breaks going out. Time spent with you guys was truly enjoyable and made my TIFR journey a little less stressful and learning more enjoyable. I would also like to thank Jagjeet, Anwesha, Toshali, Geetika, Ameya, Aditi, Suhasini, and Ayan who were a delight to be with and enjoy this journey. Within the lab, I always considered Amruta and Sravanthi as my batchmates. We have had a lot of fun both regarding science, along with other non-scientific activities. Amruta, your deep and thoughtful discussion along with precision in arguments have always been a great learning opportunity for me and I am deeply grateful for all the discussion we have had. Sravanthi, I have considered you a sister in the lab and will always cherish our endless squabbles and banters. Looking and learning from your ability to mentor along with your incisive thoughts on various topics have enriched my experience within the lab.

I am thankful to have had wonderful people within the lab including Shubha Shanbag, Dr. Jyoti Dubey, Madhushree Kamak, Dr. Souvik Modi, Amal Mathew, Keertana Venkatesh, Sri Padma Priya Boyanapalli, Sneha Hegde, Badal Singh Chauhan, Ritabhas Das, Anushka Deb, Tanushree Pathak, Swetha Nagarajan, Sahil Khichi and Sohan Seal. Extensive discussions with all of them throughout the years have been very fruitful and enjoyable. Shubha ji's work ethic is truly inspirational and I have learnt a lot from observing her work. I have closely worked with Padmapriya and Ritabhas in executing parts of this project and it was a wonderful experience interacting with both of them. Padmapriya's grit and determination is inspiring, while Ritabhas' curiosity and unabashed work was wonderful to experience. Interactions with Dr. Souvik Modi were enlightening to the broad scientific world and I have learnt a lot from him. I have worked closely with both Amal and Sneha from their first day in the lab as their rotation mentor, and it was truly a pleasure to interact with both of them since then. Amal and Keertana have been invaluable additions to the lab. I will deeply value our discussion, both scientific and otherwise. Amal's excitement about various things including sports, football and science is truly infectious. I am also deeply grateful to Sneha, who has been there for me in my final years. Her calm presence and mature thinking was irreplaceable in dealing with various stresses in my final years in TIFR. It is always a great inspiration looking at her managing TIFR outreach activities along with having a great deal of agency to start and manage a career focused club in the institute. Hanging out with Badal and Anushka along with having philosophical discussions has been deeply gratifying and I will cherish these memories in the future. I would also like to thank a number of students in the lab including Sajjita, Salik, Meet, Ashiwini, Samhita, Lipsa, Sayandeb, Dhruva, Kavya, Shirley, Somya, Rujuta, Saroja, Deepika, Tehniyyat, Sruthi, Charmi, and Akshaya for making the lab a fun environment. I would specially like to acknowledge Dhruva, Sayandeb, Sajjita, and Kavya who I have personally mentored and learnt a great deal from. I would also like to thank a short term visiting fellow in the lab Akshaya Nambiar from Ravi Manjithaya's lab. I am truly fortunate you visited while I was there. Your attitude towards science and life along with the multiple discussions we had were truly wonderful. Your infectious laughter is unforgettable. I will forever be grateful for these memories.

In the institute, I would like to thank Jitendra, Gawas, Prashant, and Gawde ji as part of the kitchen staff, without the aid of whom doing science would be an order of magnitude more difficult. I would also like to thank Veera for all the help she has provided during my tenure as a student in TIFR. No acknowledgment in DBS would be complete without thanking Boby and Parmar. Their constant efforts keep the department's numerous instruments and labs running. Their knowledge about the working various instruments is truly inspiring and I have learnt a great deal from interactions with them. I would also like to thank Ashok, Manoj, Nitin, Prakash from the accounts section; Akshata, Bhavana, Harshad, Madhukar, Rohini, Sangita, Shrikant, Suchita, Triveni, and Vishaka from the purchase section; Vijay from the low temperature facility; Shobha and Vinay from the administration section; Swapna and Shravya from the establishment section; Dilip, Pataria, Manoj and Rajesh Thakur from the central stores; Bindu and Pritam from the GS office. Your help at various stages of my journey has been essential. Interactions with some of you including Bindu, Pritam, Rajesh, Manoj, Sangita, Shrikant, Vishaka and Triveni have especially been memorable.

I would also like to express my gratitude to the entire faculty of the department for fostering an intellectually stimulating environment. The lectures, seminars, and workshops have expanded my knowledge and provided me with a platform to engage with diverse perspectives. The camaraderie among fellow students has created a supportive community that has been a source of encouragement and inspiration. Within the department, there were various people who have left a lasting experience. First and foremost, I would like to thank Ankit and Sthita for their interactions both scientific and philosophical. Discussions with them have gone a great way in refining my thinking. I would also like to thank Triveni, Suranjana, Piyal, Prateek, Janakraj, Samya, Swagata, Chaitali, Sameer, Srikanth, Sudeepa, Joshua, Praachi, Arpan and Anasuya for the feedback and perspective of science from outside the lab. The varied perspectives went a great deal and making me understand novel ways of thinking about the same problems.

This acknowledgment would be incomplete without expressing my gratitude to the countless researchers, scholars, and authors whose work has provided the foundation upon which my research is built. The body of knowledge that I have drawn upon is a testament to the collective effort of the academic community, and I am honored to contribute to its growth. I also extend my appreciation to TIFR-DAE that has provided timely financial support for my research. The support provided by the institute during COVID lockdowns was essential for my journey and I am truly grateful for it. Their investment in academic pursuits is a testament to their commitment to advancing knowledge and fostering innovation.

My heartfelt thanks go to my parents and brother for their unwavering belief in me. In addition, I would like to thank my partner, Sneha, who has always been a pillar of support for me. Your constant encouragement, understanding, and patience have been a source of strength. I am truly grateful for your love and support.

In conclusion, completing this PhD thesis would not have been possible without the guidance, support, and encouragement of countless individuals who have touched my academic journey in myriad ways. I am humbled by the collective effort that has gone into shaping this work and am excited to contribute to the ongoing discourse in this field. As I move forward in my academic and professional endeavors, I carry with me the lessons and relationships forged during this transformative journey.

Thank you all for being an integral part of this chapter in my life.