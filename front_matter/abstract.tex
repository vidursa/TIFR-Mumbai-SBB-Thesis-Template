\chapter*{Abstract}

Axonal transport is crucial for maintaining neuron structure and function, facilitated by molecular motors like UNC-104. UNC-104 utilizes microtubules in axons to move synaptic vesicle proteins (SVPs) away from the cell body. How UNC-104 distribution is maintained throughout the neuronal process is still unknown. Several mutations in UNC-104, corresponding to neurodegeneration-associated mutations in its mammalian orthologue KIF1A, lead to the depletion of UNC-104 from the neuronal cell bodies. Thus, regulating UNC-104 distribution may be critical for the neuron. In this thesis, I discuss our attempt to characterize the movement of UNC-104 \textit{in vivo} that leads to UNC-104’s observed distribution. I further elaborate on the role of two putative E3 ligases in regulating UNC-104 distribution. Experimental evidence, combined with theoretical modeling, revealed that a small fraction of processive anterograde movement of UNC-104 is counteracted by a larger fraction of non-processive diffusion to maintain UNC-104 distribution. 

To find regulators of UNC-104’s distribution, we carried out a neuron-specific RNAi screen based on the phenotype of uba-1, the E1 ubiquitin-activating enzyme, that reveals an accumulation of UNC-104 both at synapses and non-synaptic distal ends. We find that UNC-104’s synaptic and non-synaptic accumulation are independently regulated by \textit{fbxa-103} and \textit{fbxb-65} respectively. \textit{fbxb-65} regulates UNC-104 modification close to its cargo-binding PH domain, thereby regulating UNC-104’s ability to bind cargo. Dysregulated cargo binding led to the mistrafficking of SVPs away from the synapse. On the other hand, \textit{fbxa-103} likely regulates UNC-104 degradation at synapses by modifying UNC-104’s C2-like domain.

In summary, our work highlights the dual role of ubiquitin-like modifications in determining the distribution of UNC-104 within neurons. These modifications regulate UNC-104 degradation and its movement, consequently influencing cargo distribution and movement. Regulation of cargo movement in the neuronal process is essential for its correct targeting to synapses.


